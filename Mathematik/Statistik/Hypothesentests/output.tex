\documentclass[10pt]{report}
\usepackage{minted}
\usepackage{enumitem}
\setlistdepth{9}

% Sets the level up to which the numbering will occur on the headers 
\setcounter{secnumdepth}{3}

% Remove "chapter" from the chapter title
\usepackage{titlesec}
\titleformat{\chapter}[display]{\normalfont\huge\bfseries}{}{0pt}{\Huge}
\titleformat{name=\chapter,numberless}[display]
  {\normalfont\huge\bfseries}{}{0pt}{\Huge}

\setlist[itemize,1]{label=\textbullet}
\setlist[itemize,2]{label=\textbullet}
\setlist[itemize,3]{label=\textbullet}
\setlist[itemize,4]{label=\textbullet}
\setlist[itemize,5]{label=\textbullet}
\setlist[itemize,6]{label=\textbullet}
\setlist[itemize,7]{label=\textbullet}
\setlist[itemize,8]{label=\textbullet}
\setlist[itemize,9]{label=\textbullet}

\renewlist{itemize}{itemize}{9}

\usepackage{graphicx}

\usepackage{csquotes}

\usepackage[hidelinks]{hyperref}

\usepackage[T1]{fontenc}

\usepackage[most]{tcolorbox}
\definecolor{block-gray}{gray}{0.95}
\newtcolorbox{advtcolorbox}{
    %colback=block-gray,
    boxrule=0pt,
    boxsep=0pt,
    breakable,
    enhanced jigsaw,
    borderline west={4pt}{0pt}{gray},
}

% Remove paragraph indentation
\setlength{\parindent}{0pt}

\newtcbox{\pill}[1][blue]{on line,
arc=7pt,colback=#1!10!white,colframe=#1!50!black,
before upper={\rule[-3pt]{0pt}{10pt}},boxrule=1pt,
boxsep=0pt,left=6pt,right=6pt,top=2pt,bottom=2pt}

\usepackage{amsmath}
\usepackage[dvipsnames]{xcolor} % to access the named colour LightGray
\usepackage{soul}
% \usepackage{sectsty}
\definecolor{LightGray}{rgb}{0.9, 0.9, 0.9}
\definecolor{inlinecodecolor}{rgb}{0, 0.3, 0.6}
\usepackage{lmodern}

\makeatletter
\renewcommand\subparagraph{%
\@startsection{subparagraph}{5}{0pt}%
{3.25ex \@plus 1ex \@minus .2ex}{-1em}%
{\normalfont\normalsize}}
\makeatother

\makeatletter
\renewcommand\paragraph{%
\@startsection{paragraph}{4}{0pt}%
{3.25ex \@plus -1ex \@minus .2ex}{-1em}%
{\normalfont\normalsize\bfseries}}
\makeatother
\sethlcolor{yellow}
\newrobustcmd*\inlinecode[2][]{\textcolor{inlinecodecolor}{\mintinline[#1]{python}{#2}}}
\newcommand{\myparagraph}[1]{\paragraph{\textcolor{black}{#1}}\mbox{}\\}
\newcommand{\mysubparagraph}[1]{\subparagraph{\textcolor{black}{#1}}\mbox{}\\}
\begin{document}


\chapter{Hypothesentests}\label{ch:Hypothesentests}

\subsubsection{Übersicht}

+ liefert Entscheidungsregel, für welche Stichproben wird

+ $H_0$ akzeptiert

+ $H_0$ zugunsten von $H_1$ verworfen

+ kritische Bereich

+ ![[../../../z\_images/Pasted image 20221208115108.png]]

+ Menge der Stichproben

+ für die $H_0$ verworfen wird

+ Beispiele

+ Münzwurf

+ ![[../../../z\_images/Pasted image 20221208115312.png]]

+ Medikament

+ ![[../../../z\_images/Pasted image 20221208115354.png]]


\subsubsection{Ausgänge}

+ ![[../../../z\_images/Pasted image 20221208115452.png]]

+ ![[../../../z\_images/Pasted image 20221208115501.png]]

+ Level-$\alpha$-Test

+ $\alpha$ beschränkt Fehler 1. Art

+ ![[../../../z\_images/Pasted image 20221208115645.png]]

+ \hyperref[ch:P-Wert]{p-Wert}

+ Beispiele

+ Medikament

+ ![[../../../z\_images/Pasted image 20221208115842.png]]

+ ![[../../../z\_images/Pasted image 20221208115930.png]]

+ ![[../../../z\_images/Pasted image 20221208120118.png]]

+ Münzwurf

+ ![[../../../z\_images/Pasted image 20221208120318.png]]


\subsubsection{Vorgehensweise}

+ ![[../../../z\_images/Pasted image 20221208120346.png]]


\subsubsection{Einstichprobenproblem}

+ statistische Tests um Eigenschaften einer [[Stichprobe]] von [[Zufallsvariable]]n zu studieren

+ ![[../../../z\_images/Pasted image 20221208120550.png]]

+ Unterscheidung in zwei Fälle

+ \hyperref[ch:Gauß-Test]{Gauß-Test}, wenn Varianz $\sigma^2$ bekannt

+ \hyperref[ch:T-Test]{t-Test}, wenn Varianz $\sigma^2$ unbekannt


\subsubsection{Zweistichprobenproblem}

+ ![[../../../z\_images/Pasted image 20221208123736.png]]

+ Untersuchung von Hypothesenpaar für $\omega_0∈ℝ$

+ ![[../../../z\_images/Pasted image 20221208123756.png]]

+ Unterscheidung in zwei Fälle

+ \hyperref[ch:Zweistichproben-Gauß-Test]{Zweistichproben-Gauß-Test}, wenn Varianzen bekannt

+ \hyperref[ch:Zweistichproben-T-Test]{Zweistichproben-t-Test}, wenn Varianzen unbekannt


\hyperref[ch:Testen]{Testen}
\chapter{P-Wert}\label{ch:P-Wert}

\subsubsection{Motivation}

+ ![[../../../z\_images/Pasted image 20221208124621.png]]


\subsubsection{Definition}

+ ![[../../../z\_images/Pasted image 20221208124743.png]]


\subsubsection{Misconceptions}

+ ![[../../../z\_images/Pasted image 20230115143024.png]]


\subsubsection{Beispiel}

+ ![[../../../z\_images/Pasted image 20221208124847.png]]

+ ![[../../../z\_images/Pasted image 20221208125106.png]]

+ ![[../../../z\_images/Pasted image 20221208125131.png]]

+ ![[../../../z\_images/Pasted image 20221208125254.png]]

+ ![[../../../z\_images/Pasted image 20221208125336.png]]

+ ![[../../../z\_images/Pasted image 20221208125358.png]]

+ ![[../../../z\_images/Pasted image 20221208125412.png]]
\chapter{Gauß-Test}\label{ch:Gauß-Test}

\subsubsection{Definition}

+ Varianz $\sigma^2$ bekannt

+ Test von

+ ![[../../../z\_images/Pasted image 20221208122252.png]]

+ ![[../../../z\_images/Pasted image 20221208122243.png]]

+ ![[../../../z\_images/Pasted image 20221208120724.png]]

+ Es gilt

+ ![[../../../z\_images/Pasted image 20221208120847.png]]

+ ![[../../../z\_images/Pasted image 20221208120900.png]]

+ Gauß-Test mit kritischem Bereich $K$

+ ![[../../../z\_images/Pasted image 20221208121037.png]]


\subsubsection{Verwerfungsbereiche}

+ ![[../../../z\_images/Pasted image 20221208122006.png]]

+ ![[../../../z\_images/Pasted image 20221208122012.png]]

+ ![[../../../z\_images/Pasted image 20221208122123.png]]


\subsubsection{Anwendung}

+ ![[../../../z\_images/Pasted image 20221208121335.png]]

+ ![[../../../z\_images/Pasted image 20221208121351.png]]


\subsubsection{Beispiele}

+ Befüllanlage

+ ![[../../../z\_images/Pasted image 20221208123329.png]]

+ ![[../../../z\_images/Pasted image 20221208123345.png]]

+ ![[../../../z\_images/Pasted image 20221208123637.png]]
\chapter{T-Test}\label{ch:T-Test}

+ Varianz $\sigma^2$ unbekannt

+ Test von

+ ![[../../../z\_images/Pasted image 20221208122219.png]]

+ Unter $H_0$ ist

+ ![[../../../z\_images/Pasted image 20221208122314.png]]

+ wesentlich größerer Wert deutet auf Verletzung der Nullhypothese hin

+ ![[../../../z\_images/Pasted image 20221208122401.png]]

+ Es gilt

+ ![[../../../z\_images/Pasted image 20221208122452.png]]

+ t-Test mit kritischem Bereich

+ ![[../../../z\_images/Pasted image 20221208122524.png]]

+ ![[../../../z\_images/Pasted image 20221208122513.png]]


\subsubsection{Verwerfungsbereiche}

+ ![[../../../z\_images/Pasted image 20221208122541.png]]

+ ![[../../../z\_images/Pasted image 20221208122551.png]]


\subsubsection{Anwendung}

+ ![[../../../z\_images/Pasted image 20221208122720.png]]

+ ![[../../../z\_images/Pasted image 20221208122825.png]]

+ R

+ ![[../../../z\_images/Pasted image 20221208123235.png]]
\chapter{Zweistichproben-Gauß-Test}\label{ch:Zweistichproben-Gauß-Test}

\subsubsection{Definition}

+ ![[../../../z\_images/Pasted image 20221208124139.png]]

+ unter $H_0$ gilt

+ ![[../../../z\_images/Pasted image 20221208124152.png]]

+ Es folgt aufgrund der [[Unabhängigkeit von Zufallsvariablen]]

+ ![[../../../z\_images/Pasted image 20221208124324.png]]

+ ![[../../../z\_images/Pasted image 20221208124222.png]]


\subsubsection{Verwerfungsbereiche}

+ ![[../../../z\_images/Pasted image 20221208124433.png]]

+ ![[../../../z\_images/Pasted image 20221208124438.png]]
\chapter{Zweistichproben-T-Test}\label{ch:Zweistichproben-T-Test}

\chapter{Testen}\label{ch:Testen}

+ ![[../../../z\_images/Pasted image 20221215181741.png]]

+ ![[../../../z\_images/Pasted image 20221215181749.png]]

+ wird $H_0$ verworfen

+ Einfluss des Prädiktors signifikant


\subsubsection{Kritische Bereiche}

+ ![[../../../z\_images/Pasted image 20221215182234.png]]

+ ![[../../../z\_images/Pasted image 20221215182239.png]]


\subsubsection{Beispiel}

+ ![[../../../z\_images/Pasted image 20221215182927.png]]

+ ![[../../../z\_images/Pasted image 20221215183203.png]]


\subsubsection{Bestimmtheitsmaß $R^2$}

+ Gesamtstreuung der Response SST wird zerlegt in

+ durch Regressionsmodell erklärte Streuung SSR

+ Reststreuung SSE

+ ![[../../../z\_images/Pasted image 20221215183722.png]]

+ ![[../../../z\_images/Pasted image 20221215183856.png]]

+ ![[../../../z\_images/Pasted image 20221215183921.png]]


\subsubsection{Beispiel}

+ ![[../../../z\_images/Pasted image 20221215184634.png]]
\end{document}
